\large Schreibe ein Python-Script, das dafür sorgt, dass Blöcke, die man mit der rechten Maustaste anhaut, nach Vorgabe durchwechseln.
Gehe dafür wie folgt vor:
\begin{itemize}
	\item Ergänze den Code in den vorgegebenen Rahmen
	\item Variable: block\_list beinhaltet alle Blöcke durch die durch gewechselt werden soll
	\item Vorgehen:
	\begin{itemize}
		\item[1.] schreibe eine Schleife, die nicht abbricht
		\item[2.] schreibe eine weitere Schleife, die für jeden geschlagenen Block die Aktion durchführt.\begin{lstlisting}[language=Python]
		mc.events.pollBlockHits()
		\end{lstlisting}
		\item[3.] holt euch die ID des Blockes mit \texttt{getBlock}
		\item[4.] falls die ID sich in \texttt{block\_list} befindet, dann nehme den darauffolgenden Block (Achte auf die Länge der Liste)
		\item[5.] Ist das Ende der Liste erreicht, soll wieder bei dem ersten Block weiter gemacht werden.
		\item[6.] Ist der geschlagenen Block nicht in der Liste, dann soll beim ersten Block angefangen werden
		\item[7.] setze den neuen Block mit \texttt{setBlock}
	\end{itemize}
\end{itemize}
Easy:
\begin{itemize}
	\item Ergänze den Code in dem vorgegebenen Rahmen
	\item Variable: block\_list beinhaltet alle Blöcke durch die durch gewechselt werden soll
	\item Vorgehen:
	\begin{itemize}
		\item[1.] lege eine Variable an, die die Listen-Position des aktuellen Steines zu speichern
		\item[2.] schreibe eine Schleife, die nicht abbricht
		\item[3.] schreibe eine weitere Schleife, die für jeden geschlagenen Block die Aktion durchführt.\begin{lstlisting}[language=Python]
		mc.events.pollBlockHits()
		\end{lstlisting}
		\item[4.] ersetze den geschlagenen Block durch den Block auf den die Variable zeigt
		\item[5.] Wenn die Variable das Ende der Liste erreicht hat, soll sie wieder bei 0 beginnen, ansonsten erhöht werden
	\end{itemize}
\end{itemize}