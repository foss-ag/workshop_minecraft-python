Erweitere den ersten Aufgabenteil wie folgt:
\begin{itemize}
	\item Wird ein Block geschlagen, soll er nicht einfach durch den aktuellen Block in der Liste ersetzt werden, sondern abhängig vom geschlagenen Block behandelt werden. Ist der geschlagene Block nicht in der Liste vorhanden, so wird er durch den ersten Block der Liste ersetzt. Befindet sich der Block in der Liste, so wird er durch den nachfolgenden Block in der Liste ersetzt.
	
	\item Um herauszufinden, um welche Art von Block es sich handelt, kannst du die Funktion\\ \texttt{mc.getBlock(x\_pos, y\_pos, z\_pos)} verwenden.
	
	\item Hast du das Ende der Liste erreicht, fange wieder beim ersten Element an.
\end{itemize}