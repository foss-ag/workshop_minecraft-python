\large Schreibe ein Python-Script, das einen zufälligen Parcour über das Lavabecken generiert.
\begin{itemize}
	\item Um das Lavabecken und den restlichen Rahmen zu generieren, führe das Initiliasierungsscript im Terminal aus.
	\begin{lstlisting}[language=sh]
python lava_runner_initializer.py
	\end{lstlisting}
	\item Binde die Minecraft-Bibliothek ein, um mit dem Spiel zu kommunizieren.
	\begin{lstlisting}[language=Python]
import mcpi.minecraft as minecraft
	\end{lstlisting}
	
	\item Öffne die Datei \texttt{lava\_runner.py} mit dem Editor. Diese Datei enthält bereits Code und muss lediglich im gekennzeichneten Bereich erweitert werden.
	
	\item Die Variablen \texttt{x}, \texttt{y}, \texttt{z} beschreiben die Position des zu setzenden Steins, \texttt{x\_boundary} und \texttt{z\_boundary} sind die Grenzen der Arena und markieren das Ende des Pfades.
	
	\item Schreibe eine Schleife, die abbricht, sobald die generierte Position über die Grenze \texttt{x\_boundary} oder \texttt{z\_boundary} tritt.

	\item Setze mit \texttt{setBlock(x\_pos, y\_pos, z\_pos, block)}  einen Eisblock (\texttt{block.ICE}) an die zuletzt generierte Position.

	\item Berechne eine zufällige Position, an der der nächste Block platziert werden soll. Nutze dafür die folgende Funktion:
		\begin{lstlisting}[language=Python]
random.randint(lower_bound, upper_bound)
		\end{lstlisting}
		
	\item Frage mit einer \texttt{if}-Bedingung die generierte Position ab und überprüfe, ob diese von Steve erreichbar ist (siehe Abbildung). Kann die neue Position nicht erreicht werden, breche den aktuellen Schleifendurchlauf mit \texttt{continue} ab, falls doch, ersetze die alte Position durch die gerade generierte Position.
\end{itemize}