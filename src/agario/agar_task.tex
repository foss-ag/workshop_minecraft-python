\textbf{\large agar.io :}\\
Werde die größte Zelle im Universum!!! Doch iergendwer hat dir Code geklaut. Bevor du spielen kannst, musst du erst mal ein paar Schritte machen.\\\\
\textbf{Schritt 1: Kollisionen}\\
Damit du die anderen Zellen aufessen kannst, musst du überprüfen ob du mit jemanden zusammen gestoßen bist.
\begin{itemize}
	\item Hole dir nacheinander jede Zelle aus der Liste \textit{cell\_list}
	\item Überprüfe mit der Funktion \textit{check\_bigger(cell.x, cell.y)} ob du größer bist als die andere Zelle.
	\item Ist das der Fall, dann:
	\begin{itemize}
		\item Erhöhe deine Masse \textit{self.mass} um 0.5
		\item Entferne die Zelle \textit{cell} aus der Liste \textit{cell\_list} mit dem Befehl \textit{cell\_list.remove(cell)}
	\end{itemize}
	\end{itemize}
\textbf{Schritt 2: Neue Zellen erzeugen}\\
Damit du immer weiter wachsen kannst, müssen immer neue Zellen erzeugt werden. Dann kannst du immer weiter essen und weiter wachsen.
\begin{itemize}
	\item Lege eine Variable \textit{i} mit dem Wert \textbf{0} an
	\item Programmiere eine Schleife die erst dann abbricht, wenn die Variable \textit{i} den Wert \textbf{numOfCells} erreicht hat
	\item Erzeuge eine neue Zelle mit dem Befehl \textit{cell = Cell(surface)}
	\item Füge die Zelle zu der Liste \textit{cell\_list}, durch den Befehl \textit{cell\_list.append}, hinzu
	\item Nicht vergessen \textit{i} zu erhöhen ;)
\end{itemize}
\textbf{Schritt 3: Der Welt beitreten}\\
Damit du in die Welt rein kommst und nicht alleine bist, müssen diese drei Sachen gemacht werden.
\begin{itemize}
	\item Als erstes brauchst du die Kamera, die deine Zelle verfolgt. Benutze dazu den Befehl \textit{camera = Camera()}
	\item Erzeuge nun einen \textit{Player}. Benutze dazu diesen Befehl \textit{blob = Player(surface, "Name")}. ersetzt \textbf{Name} durch deinen Namen.
	\item Erzeuge jetzt beliebig viele andere Zellen. Fang erst mal mit \textbf{2000} an. Benutze dafür den Befehl \textit{spawn\_cells(2000)}
\end{itemize}
\textbf{Schritt 4: Teilen}\\
Nun muss die Zelle noch teilbar sein, damit schnellere Gegner eingefangen werden können.
\begin{itemize}
	\item Schreibe eine While-Schleife, die nicht endet
	\item In diese muss der Befehl \textit{clock.tick(70)} damit sich das Spiel verändert
	\item Außerdem muss dort eine For-Schleife rein, die jedes Event \textit{e} der Liste, die von \textit{pygame.event.get()} geholt werden kann, folgendes tut:
	\begin{itemize}
		\item Du brauchst eine If-Bedingung, die prüft ob \textbf{e.type = = pygame.KEYDOWN} stimmt
		\begin{itemize}
			\item In der If-Bedingung brauchst du drei weitere If-Bedingungen
			\item[1.] Prüfe ob \textbf{e.key = = pygame.K\_ESCAPE} gilt:
			\begin{itemize}
				\item Beende das Spiel mit folgenden beiden Befehlen: \textit{pygmae.quit()} und \textit{quit()}			
			\end{itemize}	
			\item[2.] Prüfe ob \textbf{e.key = = pygame.K\_SPACE} gilt:
			\begin{itemize}
				\item Teile deine Blase mit dem Befehl \textit{blob.split()}
			\end{itemize}					 
			\item[3.] Prüfe ob \textbf{e.key = = pygame.K\_w} gilt:
			\begin{itemize}
				\item Füttere einen anderen Spieler, indem du die Funktion \textit{blob.feed()} benutzt
			\end{itemize}
		\end{itemize}
		\item Du brauchst in der For-Schleife eine weitere If-Bedingung, die \textbf{e.type = = pygame.QUIT} prüft:
		\begin{itemize}
			\item Beende darin das Spiel mit folgenden Befehlen: \textit{pygame.quit()} und \textit{quit()}
		\end{itemize}
	\end{itemize}	 
\end{itemize}