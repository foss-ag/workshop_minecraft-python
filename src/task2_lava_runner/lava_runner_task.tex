\large Schreibe ein Python-Script, das einen zufälligen Parcour über das Lavabecken generiert.
\begin{itemize}
	\item Um das Lavabecken und den restlichen Rahmen zu generieren, führe das Initiliasierungsscript im Terminal aus.
	\begin{lstlisting}[language=sh]
python lava_runner_initializer.py
	\end{lstlisting}
	\item Binde die Minecraft-Bibliothek ein, um mit dem Spiel zu kommunizieren.
	\begin{lstlisting}[language=Python]
import mcpi.minecraft as minecraft
	\end{lstlisting}
	
	\item Öffne die Datei \texttt{lava\_runner.py} mit dem Editor. Diese Datei enthält bereits Code und muss lediglich im gekennzeichneten Bereich erweitert werden.
	
	\item Parameter: Die Koordinaten x, y, z beinhalten die Position des zu setzenden Steins
	\item Variablen: Die Variablen x\_boundary, z\_boundary sind die Grenzen der Arena und markieren das Ende des Pfades
	\item Aufgaben:
	\begin{itemize}
		\item[1.] Schreibe eine Schleife, die abbricht, wenn die berechnete Steinposition größer als die x\_boundary oder z\_boundary ist
		%\item[1.] Schreibt eine Schleife, die läuft, solange die berechnete Steinposition kleiner als die x\_boundary und z\_boundary ist
		\item[2.] Berechne mit der \begin{lstlisting}[language=Python]
		random.randint(lower_bound, upper_bound)
		\end{lstlisting} Funktion neue x, y, z Positionen
		\item[3.] Fange mit einer If-Bedingung die Positionen ab, die vom Spieler nicht erreicht werden können
		\item[4.] ersetze die alte Steinposition durch die gerade berechnete Steinposition
		\item[5.] setze mit der \texttt{setBlock} Methode einen \begin{lstlisting}[language=Python]
		block.GLOWSTONE_BLOCK
		\end{lstlisting} an die aktuelle Steinposition
	\end{itemize}
\end{itemize}