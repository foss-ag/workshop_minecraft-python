\large Vervollständige den Code, sodass das Spiel funktioniert. Damit die Melonen verschwinden muss eine Schleife ergänzt werden:
\begin{itemize}
	\item Ergänze den Code nur in dem gekennzeichneten Rahmen.
	
	\item Die Methode \texttt{check\_tapped} überprüft während des Spieles, ob eine Melone geschlagen wurde.
	
	\item Schreibe innerhalb der \texttt{while}-Schleife eine \texttt{for}-Schleife, die für jeden geschlagenen Block den Schleifenrumpf ausführt. Um die Liste der angeschlagenen Blöcke zu erhalten verwende die Funktion
	\begin{lstlisting}
mc.events.pollBlockHits()
	\end{lstlisting}
		
	\item Überprüfe durch eine if-Bedingung, ob der geschlagene Block eine Melone war.
	
	\item Lege dir dazu als erstes eine Variable an, in der du den Block speicherst. Den Block bekommst du durch
	\begin{lstlisting}
mc.getBlock(x_pos, y_pos, z_pos)
	\end{lstlisting}
	
	\item Die ID von Melonen ist die 103.
	
	\item Wenn eine Melone geschlagen wurde, ersetze diese durch Luft. Nutze dafür die Funktion
	\begin{lstlisting}
mc.setBlock(x_pos, y_pos, z_pos, 0)
	\end{lstlisting}
	Zähle dannach sofort den Wert von \texttt{count} hoch, damit überprüft werden kann, ob alle Melonen erwischt wurden.
	
	\item Damit die Überprüfung der Zeit ordentlich funktioniert, setze in der if-Bedingung noch den wert von \texttt{x\_old} auf den Wert von \texttt{x\_cur}
\end{itemize}