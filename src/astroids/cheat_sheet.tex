\documentclass{article}

\usepackage[english]{babel} % required to compile for windows
\usepackage[utf8]{inputenc}
\usepackage[T1]{fontenc}
\usepackage{geometry}
\usepackage{listings}
\usepackage{xcolor}
\lstset{ %
	frame=single,	                   % adds a frame around the code
	numbers=left,                    % where to put the line-numbers; possible values are (none, left, right)
	numbersep=5pt,                   % how far the line-numbers are from the code
	numberstyle=\tiny\color{gray}, % the style that is used for the line-numbers
	showstringspaces=false
}
\usepackage{enumitem}
\setlist[itemize]{leftmargin=*}
\usepackage{fancyhdr} % header
\geometry{a4paper, left=2cm, right=2cm, top=3cm, bottom=4cm} 

% header
\pagestyle{fancy}
\fancyhead{} % clear header
\fancyfoot{} % clear footer
\setlength{\headheight}{50pt}

\chead{{\large Schnupperuni Informatik}}

\lhead{FOSS-AG}
\rhead{15-19.10.2018}
\cfoot{-\thepage-}

\begin{document}
	\begin{center}
		{\huge PyGame Cheat Sheet}
	\end{center}

	\section{nützliche Befehle}
		Das PyGame Modul enthält viele Funktionen, die man früher oder später brauchen wird. Hier ist die Funktion, die wir für das Astroid Spiel brauchen.\\
		\begin{itemize}
			\item Taste gedrückt:
			\item \begin{lstlisting}[language=Python]
pressed = pygame.key.get_pressed()
		\end{lstlisting}
			\item Beispielabfrage: pressed[pygame.K\_w] (Taste \textit{w} wurde gedrückt
		\end{itemize}
		Um noch weitere Tasten abfragen zu können ist hier eine Liste mit allen Schlüsseln der Tasten.\\
		\begin{tabular}{l | r}
		Name & Abkürzung \\
		$[a - z]$ & K\_[a - z] \\
		$[0 - 9]$ & K\_[0 - 9] \\
		oben Pfeil & K\_UP \\
		unten Pfeil & K\_DOWN \\
		links Pfeil & K\_LEFT \\
		rechts Pfeil & K\_RIGHT \\
		F[1 - 12] & K\_F[1 - 12] \\
		\end{tabular}
\end{document}
