\textbf{\large Raumfahrer Training 101:}\\
Du bist mit deinem Raumschiff mitten im Weltall unterwegs. Bereit neue Planeten zu entdecken. Doch irgendwo im Andromeda-Nebel versagen deine Motoren. Niemand weit und breit, der dich zum nächsten Planeten bringen kann. Nicht mal Asteroiden sind zu sehen.\\\\
\textbf{Schritt 1: Asteroiden erscheinen lassen}\\
Versuche zuerst die Asteroiden erscheinen zu lassen, dann ist es im Weltraum nicht mehr so leer.
\begin{itemize}
	\item Prüfe ob der Asteroiden Timer (\texttt{state.astroid\_timer}) abgelaufen ist
	\begin{itemize}
		\item \texttt{state.astroid\_timer == 0}
	\end{itemize}
	\item Ist dieser abgelaufen, erzeuge einen Asteroiden mithilfe des Befehles \texttt{Astroid.create\_astroid()}
	\item Setze den \texttt{state.astroid\_timer} neu, auf den Wert \textbf{100 - (astroids\_faster * 2)}. Nutze dafür die Funktion \texttt{state.set\_astroid\_timer(t)}
	\item Ändere den Wert von \texttt{state.astroids\_faster}
	\begin{itemize}
		\item Gilt \texttt{state.astroids\_faster} $\geq$ 35, dann setze den Wert wieder auf 35
		\item ansonsten erhöhe den Wert von \texttt{state.astroid\_timer} um 5
		\item Dafür steht die Funktion \texttt{state.set\_astroids\_faster(af)} zur Verfügung.
	\end{itemize}
\end{itemize}
\textbf{Schritt 2: Kollisionen}\\
Jetzt sind wieder viele Asteroiden unterwegs, aber ohne Kollisionen ist das Ganze langweilig.
\begin{itemize}
	\item Überprüfe mithilfe des Befehls \texttt{player\_astroid\_collision\_check(astroid, astroid\_rect, player\_rect)} ob eine Kollision vorliegt.
	\item Ist das der Fall unterscheide 2 Fälle:
	\begin{itemize}
		\item Hat der Spieler noch Leben, dann verringere die Lebensanzahl
		\item Ansonsten beende das Spiel, in dem du die Funktion \texttt{state.set\_done()} aufrufst.
	\end{itemize}
\end{itemize}
\textbf{Schritt 3: Asteroiden treffen}\\
Nun kommen Asteroiden auf dich zu, aber die Zielsysteme funktionieren scheinbar nicht. Keiner der Schüsse scheint zu treffen.
\begin{itemize}
	\item Für jeden Schuss in der Liste \texttt{state.shots} muss geguckt werden ob der aktuelle Asteroid getroffen wurde.
	\begin{itemize}
		\item Speichere die Ausmaße des aktuellen Asteroiden in der Variablen \texttt{bullet\_rect} ab. Die Ausmaße bekommst du durch den Befehl \texttt{bullet.get\_rect\_bullet()}
		\item Überprüfe mit dem Befehl \texttt{check\_bullet\_astroid\_hit(bullet, bullet\_rect, astroid\_rect)} ob der Asteroid getroffen wurde.
		\begin{itemize}
			\item Wurde der Asteroid getroffen, dann erhöhe den Wert von Zähler für die Treffer um 1 (\texttt{state.increment\_num\_hits()})
			\item Überprüfe, ob der Wert \texttt{astroid.hit\_count} gleich 1 ist.
			\begin{itemize}
				\item Gilt dies, dann entferne den Asteroiden mit \texttt{state.remove\_astroid(astroid)}.
				\item Ansonsten erhöhe den Zähler für die Treffer mit \texttt{astroid.increment\_hit\_count()}.
			\end{itemize}
		\end{itemize}
	\end{itemize}
\end{itemize}
\textbf{Schritt 4: Bewegen}\\
Nun bewegen wir uns wieder durch das Weltall, aber wie sollen wir zu unserem Ziel kommen, wenn wir nicht ausweichen können?
\begin{itemize}
	\item Überlege dir Tasten mit denen du das Raumschiff steuern willst.
	\item Finde heraus welche Taste gedrückt wurde
	\begin{itemize}
		\item \texttt{pressed = pygame.key.get\_pressed()}
		\item \texttt{if pressed[pygame.K\_DOWN]} (Pfeiltaste nach unten)
	\end{itemize}
	\item Schreibe eine if-Bedingung für jede Bewegungstaste
	\item Bewege den Spieler mit player.move(x, y)
	\begin{itemize}
		\item \texttt{x} ist verantwortlich für einen Flug nach links oder rechts
		\item \texttt{y} ist verantwortlich für einen Flug nach oben oder unten
		\item Verändere den Wert um den Wert der Variablen \texttt{player\_speed}
		\item Soll nur einer der beiden Werte verändert werden, setze den anderen Wert auf 0.
	\end{itemize}	 
\end{itemize}