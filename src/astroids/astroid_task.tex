\textbf{\large Raumfahrer Training 101:}\\
Du bist mit deinem Raumschiff mitten im Weltall unterwegs. Bereit neue Planeten zu entdecken. Doch irgendwo im Andromeda-Nebel versagen deine Motoren. Niemand weit und breit, der dich zum nächsten Planeten bringen kann. Nicht mal Asteroiden sind zu sehen.\\\\
\textbf{Schritt 1: Asteroiden erscheinen lassen}\\
Versuche zuerst die Asteroiden erscheinen zu lassen, dann ist es im Weltraum nicht mehr so leer.
\begin{itemize}
	\item Prüfe ob der Asteroiden Timer (\textit{state.astroid\_timer}) abgelaufen ist
	\begin{itemize}
		\item \textit{state.astroid\_timer == 0}
	\end{itemize}
	\item Ist dieser abgelaufen, erzeuge einen Asteroiden mithilfe des Befehles \textit{Astroid.create\_astroid()}
	\item Setze den \textit{state.astroid\_timer} neu, auf den Wert \textbf{100 - (astroids\_faster * 2)}. Nutze dafür die Funktion \textit{state.set\_astroid\_timer(t)}
	\item Ändere den Wert von \textit{state.astroids\_faster}
	\begin{itemize}
		\item Gilt \textit{state.astroids\_faster} $\geq$ 35, dann setze den Wert wieder auf 35
		\item ansonsten erhöhe den Wert von \textit{state.astroid\_timer} um 5
		\item Dafür steht die Funktion \textit{state.set\_astroids\_faster(af)} zur Verfügung.
	\end{itemize}
\end{itemize}
\textbf{Schritt 2: Kollisionen}\\
Jetzt sind wieder viele Asteroiden unterwegs, aber ohne Kollisionen ist das Ganze langweilig.
\begin{itemize}
	\item Überprüfe mithilfe des Befehls \textit{player\_astroid\_collision\_check(astroid, astroid\_rect, player\_rect)} ob eine Kollision vorliegt.
	\item Ist das der Fall unterscheide 2 Fälle:
	\begin{itemize}
		\item Hat der Spieler noch Leben, dann verringere die Lebensanzahl
		\item Ansonsten beende das Spiel, in dem du die Funktion \textit{state.set\_done()} aufrufst.
	\end{itemize}
\end{itemize}
\textbf{Schritt 3: Asteroiden treffen}\\
Nun kommen Asteroiden auf dich zu, aber die Zielsysteme funktionieren scheinbar nicht. Keiner der Schüsse scheint zu treffen.
\begin{itemize}
	\item Für jeden Schuss in der Liste \textit{state.shots} muss geguckt werden ob der aktuelle Asteroid getroffen wurde.
	\begin{itemize}
		\item Speichere die Ausmaße des aktuellen Asteroiden in der Variablen \textit{bullet\_rect} ab. Die Ausmaße bekommst du durch den Befehl \textit{bullet.get\_rect\_bullet()}
		\item Überprüfe mit dem Befehl \textit{check\_bullet\_astroid\_hit(bullet, bullet\_rect, astroid\_rect)} ob der Asteroid getroffen wurde.
		\begin{itemize}
			\item Wurde der Asteroid getroffen, dann erhöhe den Wert von Zähler für die Treffer um 1 (\textit{state.increment\_num\_hits()})
			\item Überprüfe, ob der Wert \textit{astroid.hit\_count} gleich 1 ist.
			\begin{itemize}
				\item Gilt dies, dann entferne den Asteroiden mit \textit{state.remove\_astroid(astroid)}.
				\item Ansonsten erhöhe den Zähler für die Treffer mit \textit{astroid.increment\_hit\_count()}.
			\end{itemize}
		\end{itemize}
	\end{itemize}
\end{itemize}
\textbf{Schritt 4: Bewegen}\\
Nun bewegen wir uns wieder durch das Weltall, aber wie sollen wir zu unserem Ziel kommen, wenn wir nicht ausweichen können?
\begin{itemize}
	\item Überlege dir Tasten mit denen du das Raumschiff steuern willst
	\item Finde heraus welche Taste gedrückt wurde
	\begin{itemize}
		\item pressed = pygame.key.get\_pressed()
		\item if pressed[pygame.K\_DOWN] (Pfeiltaste nach unten)
	\end{itemize}
	\item Schreibe eine If-Bedingung für jede Bewegungstaste
	\item Verändere die Variablen \textit{x} und \textit{y}
	\begin{itemize}
		\item \textit{x} ist verantwortlich für einen Flug nach links oder rechts
		\item \textit{y} ist verantwortlich für einen Flug nach oben oder unten
		\item Verändere den Wert um den Wert der Variablen \textit{player\_speed}
	\end{itemize}	 
\end{itemize}