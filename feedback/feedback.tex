\documentclass[
  % Babel language, also used to load translations
  german,
  % Use A4 paper size, you can change this to eg. letterpaper if you need
  % the letter format. The normal methods to modify the paper size should
  % be picked up by SDAPS automatically.
  % a4paper, % setting this might break the example scan unfortunately
  % letterpaper
  %
  % If you need it, you can add a custom barcode at the center
  %globalid=SDAPS,
  %
  % And the following adds a per sheet barcode at the bottom left
  %print_questionnaire_id,
  %
  % You can choose between twoside and oneside. twoside is the default, and
  % requires the document to be printed and scanned in duplex mode.
  %oneside,
  %
  % With SDAPS 1.1.6 and newer you can choose the mode used when recognizing
  % checkboxes. valid modes are "checkcorrect" (default), "check" and
  % "fill".
  %checkmode=checkcorrect,
  %
  % The following options make sense so that we can get a better feel for the
  % final look.
  pagemark,
  stamp]{sdapsclassic}
\usepackage[utf8]{inputenc}
% For demonstration purposes
\usepackage{multicol}

\author{FOSS-AG}
\title{Von Schlangen und Himbeerkuchen - Programmieren in der Welt von Minecraft}

\begin{document}
  % Everything you do should be done inside the questionnaire environment.

  % If you don't like the default text at the beginning of each questionnaire
  % you can remove it with the optional [noinfo] parameter for the environment 
  \begin{questionnaire}
    \section{Vorkenntnisse}

	\begin{choicequestion}[cols=2]{Hast du bereits Programmiervorkenntnisse?}
		\choiceitem{Ja}
		\choiceitem{Nein}
	\end{choicequestion}

    \singlemark{Falls ja, wie würdest du deine Vorkenntnisse einschätzen?}{sehr gering}{sehr hoch}
    
    \textbox{2cm}{Anmerkungen zu den Vorkenntnissen?}
    
    \section{Aufgabenstellung}
    \begin{markgroup}{Wie verständlich sind die einzelnen Aufgabenstellungen?}
    	\markline{Blumenpfad}{unverständlich}{sehr verständlich}
    	\markline{Shuffle Block 1}{unverständlich}{sehr verständlich}
    	\markline{Shuffle Block 2}{unverständlich}{sehr verständlich}
    	\markline{Lava Runner}{unverständlich}{sehr verständlich}
    \end{markgroup}
    
    \begin{markgroup}{Für wie schwer empfindest du die einzelnen Aufgaben?}
    	\markline{Blumenpfad}{zu einfach}{zu schwer}
    	\markline{Shuffle Block 1}{zu einfach}{zu schwer}
    	\markline{Shuffle Block 2}{zu einfach}{zu schwer}
    	\markline{Lava Runner}{zu einfach}{zu schwer}
	\end{markgroup}

	\textbox{2cm}{Anmerkungen zu den Aufgaben?}
	
	\section{Sonstiges}
	\begin{choicequestion}[cols=2]{Hat dir der Workshop gefallen?}
		\choiceitem{Ja}
		\choiceitem{Nein}
	\end{choicequestion}

	\textbox{2cm}{Was sollten wir beim nächsten Mal anders machen?}
  \end{questionnaire}
\end{document}

