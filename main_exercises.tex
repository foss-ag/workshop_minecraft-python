\documentclass{article}

\usepackage[english]{babel} % required to compile for windows
\usepackage[utf8]{inputenc}
\usepackage[T1]{fontenc}
\usepackage{geometry}
\usepackage{listings}
\usepackage{xcolor}
\lstset{ %
	frame=single,	                   % adds a frame around the code
	numbers=none,                    % where to put the line-numbers; possible values are (none, left, right)
	numbersep=5pt,                   % how far the line-numbers are from the code
	numberstyle=\tiny\color{gray}, % the style that is used for the line-numbers
	showstringspaces=false
}
\usepackage{enumitem}
\setlist[itemize]{leftmargin=*}
\usepackage{graphicx}
\usepackage{fancyhdr} % header
\geometry{a4paper, left=2cm, right=2cm, top=3cm, bottom=4cm} 

% header
\pagestyle{fancy}
\fancyhead{} % clear header
\fancyfoot{} % clear footer
\setlength{\headheight}{50pt}

\chead{{\large Schnupperuni Informatik}}

\lhead{FOSS-AG}
\rhead{23. - 26.10.2017}
\cfoot{-\thepage-}

\begin{document}
	\begin{center}
		\huge Von Schlangen und Himbeerkuchen:\\Programmieren in der Welt von Minecraft
	\end{center}
	\section{Einführung}
		\large Schreibe ein Python-Script, das dafür sorgt, dass stets Blumen auf dem Block generiert werden, auf dem Steve sich befindet.
Gehe dafür wie folgt vor:
\begin{itemize}
	\item Binde die Minecraft-Bibliothek ein, um mit dem Spiel zu kommunizieren.
	\begin{lstlisting}[language=Python]
import mcpi.minecraft as minecraft
	\end{lstlisting}
	
	\item Baue die Verbindung zum Spiel auf.
	\begin{lstlisting}[language=Python]
mc = minecraft.Minecraft.create()
	\end{lstlisting}
	Das Objekt \textbf{mc} stellt nun die Kommunikationschnittstelle mit dem Spiel dar.
	
	\item Speichere dir die Block-ID der Blume (38) in einer Variable zwischen. Das erleichtert das Programmieren und macht den Code einfacher zu lesen.
	
	\item Verwende eine \textbf{while}-Schleife für den Hauptteil deines Scripts. Da das Platzieren der Blumen solange stattfinden soll, wie das Spiel läuft, wird eine Schleife benötigt, die nur durch Beenden des Programms abgebrochen werden kann.
	
	\item Frage innerhalb der \textbf{while}-Schleife die aktuelle Position von Steve ab. Diese benötigen wir, um an genau dieser Stelle die Blumen platzieren zu können.
	
	\item Platziere zu guter letzt an Steves aktueller Position die Blumen. 
\end{itemize}
	\section{Shuffle Block}
		\subsection{}
		\large Schreibe ein Python-Script, das dafür sorgt, dass Blöcke, die man mit der rechten Maustaste anhaut, nach Vorgabe durchwechseln.
Gehe dafür wie folgt vor:
\begin{itemize}
	\item Ergänze den Code in den vorgegebenen Rahmen
	\item Variable: block\_list beinhaltet alle Blöcke durch die durch gewechselt werden soll
	\item Vorgehen:
	\begin{itemize}
		\item[1.] schreibe eine Schleife, die nicht abbricht
		\item[2.] schreibe eine weitere Schleife, die für jeden geschlagenen Block die Aktion durchführt.\begin{lstlisting}[language=Python]
		mc.events.pollBlockHits()
		\end{lstlisting}
		\item[3.] holt euch die ID des Blockes mit \texttt{getBlock}
		\item[4.] falls die ID sich in \texttt{block\_list} befindet, dann nehme den darauffolgenden Block (Achte auf die Länge der Liste)
		\item[5.] Ist das Ende der Liste erreicht, soll wieder bei dem ersten Block weiter gemacht werden.
		\item[6.] Ist der geschlagenen Block nicht in der Liste, dann soll beim ersten Block angefangen werden
		\item[7.] setze den neuen Block mit \texttt{setBlock}
	\end{itemize}
\end{itemize}
Easy:
\begin{itemize}
	\item Ergänze den Code in dem vorgegebenen Rahmen
	\item Variable: block\_list beinhaltet alle Blöcke durch die durch gewechselt werden soll
	\item Vorgehen:
	\begin{itemize}
		\item[1.] lege eine Variable an, die die Listen-Position des aktuellen Steines zu speichern
		\item[2.] schreibe eine Schleife, die nicht abbricht
		\item[3.] schreibe eine weitere Schleife, die für jeden geschlagenen Block die Aktion durchführt.\begin{lstlisting}[language=Python]
		mc.events.pollBlockHits()
		\end{lstlisting}
		\item[4.] ersetze den geschlagenen Block durch den Block auf den die Variable zeigt
		\item[5.] Wenn die Variable das Ende der Liste erreicht hat, soll sie wieder bei 0 beginnen, ansonsten erhöht werden
	\end{itemize}
\end{itemize}
		\subsection{}
		Erweitere den ersten Aufgabenteil wie folgt:
\begin{itemize}
	\item Wird ein Block geschlagen, soll er nicht einfach durch den aktuellen Block in der Liste ersetzt werden, sondern abhängig vom geschlagenen Block behandelt werden. Ist der geschlagene Block nicht in der Liste vorhanden, so wird er durch den ersten Block der Liste ersetzt. Befindet sich der Block in der Liste, so wird er durch den nachfolgenden Block in der Liste ersetzt.
	
	\item Um herauszufinden, um welche Art von Block es sich handelt, kannst du die Funktion\\ \texttt{mc.getBlock(x\_pos, y\_pos, z\_pos)} verwenden.
	
	\item Hast du das Ende der Liste erreicht, fange wieder beim ersten Element an.
\end{itemize}
	\section{Lava Runner}
		\large Schreibe ein Python-Script, das einen zufälligen Parcour über das Lavabecken generiert.
\begin{itemize}
	\item Um das Lavabecken und den restlichen Rahmen zu generieren, führe das Initiliasierungsscript im Terminal aus.
	\begin{lstlisting}[language=sh]
python lava_runner_initializer.py
	\end{lstlisting}
	\item Binde die Minecraft-Bibliothek ein, um mit dem Spiel zu kommunizieren.
	\begin{lstlisting}[language=Python]
import mcpi.minecraft as minecraft
	\end{lstlisting}
	
	\item Öffne die Datei \texttt{lava\_runner.py} mit dem Editor. Diese Datei enthält bereits Code und muss lediglich im gekennzeichneten Bereich erweitert werden.
	
	\item Die Variablen \texttt{x}, \texttt{y}, \texttt{z} beschreiben die Position des zu setzenden Steins, \texttt{x\_boundary} und \texttt{z\_boundary} sind die Grenzen der Arena und markieren das Ende des Pfades.
	
	\item Schreibe eine Schleife, die abbricht, sobald die generierte Position über die Grenze \texttt{x\_boundary} oder \texttt{z\_boundary} tritt.

	\item Setze mit \texttt{setBlock(x\_pos, y\_pos, z\_pos, block)}  einen Eisblock (\texttt{block.ICE}) an die zuletzt generierte Position.

	\item Berechne eine zufällige Position, an der der nächste Block platziert werden soll. Nutze dafür die folgende Funktion:
		\begin{lstlisting}[language=Python]
random.randint(lower_bound, upper_bound)
		\end{lstlisting}
	Dabei soll die nächste Position in der y-Achse maximal 1 größer sein als die vorherige Position. Für die x- und z-Achse gilt, dass die nächste Position mindestens 1 und maximal 2 größer sein soll als die vorherige.
		
	\item Frage mit einer \texttt{if}-Bedingung die generierte Position ab und überprüfe, ob diese von Steve erreichbar ist (siehe Abbildung). Kann die neue Position nicht erreicht werden, breche den aktuellen Schleifendurchlauf mit \texttt{continue} ab, falls doch, ersetze die alte Position durch die gerade generierte Position.
\end{itemize}
\begin{figure}
\centering
\includegraphics[scale=0.25]{src/lava_runner/res/1layer.png}
\caption{Das schwarz markierte Feld stellt die Position des vorherigen Blocks dar, die roten Felder sind nicht erreichbar, weshalb ein Block verworfen werden soll, falls er zufällig auf diesem Feld platziert werden soll. Die zu erreichenden Blöcke gilt auch für die Ebene in der sich der Spieler befindet.}
\end{figure}
\end{document}\grid
