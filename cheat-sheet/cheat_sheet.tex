\documentclass{article}

\usepackage[english]{babel} % required to compile for windows
\usepackage[utf8]{inputenc}
\usepackage[T1]{fontenc}
\usepackage{geometry}
\usepackage{listings}
\usepackage{fancyhdr} % header
\geometry{a4paper, left=2cm, right=2cm, top=3cm, bottom=4cm} 

% header
\pagestyle{fancy}
\fancyhead{} % clear header
\fancyfoot{} % clear footer
\setlength{\headheight}{50pt}

\chead{{\large Schnupperuni Informatik}}

\lhead{FOSS-AG}
\rhead{23-26.10.2017}
\cfoot{-\thepage-}

\begin{document}
	\begin{center}
		{\huge Python Cheat Sheet}
	\end{center}

	\section{Variablen}
		In Variablen lassen sich beliebige Werte, aber auch Ergebnisse von Berechnung und Funktionen abspeichern. Allgemeines Schema:
		\begin{lstlisting}[language=Python]
variable_name = value
		\end{lstlisting}
		Beispiele:
		\begin{lstlisting}[language=Python]
text = "Hello World"
x = 3
a = x + 1
position = mc.Player.getPos()
		\end{lstlisting}
	
	\section{Bedingungen}
	Manchmal möchte man einen bestimmten Codeabschnitt nur dann ausführen, wenn gewisse Bedingungen gelten. Genau für diese Fälle verwendet man \texttt{if-}Abfragen. Ist die Bedingung wahr, so wird der Abschnitt ausgeführt, andernfalls wird er einfach übersprungen oder der dazugehörige \texttt{else}-Block ausgeführt. Allgemeines Schema:
	\begin{lstlisting}[language=Python]
if CONDITION:
  # your code here#
	\end{lstlisting}
	Beispiele:
	\begin{lstlisting}[language=Python]
if x == 10:
  x = x - 5

if x > 0 and x < 3:
  x = 10
else:
  x = 20
	\end{lstlisting}
	
	\section{Schleifen}
	Mit Hilfe von Schleifen kann ein bestimmter Codeabschnitt mehrfach ausgeführt werden oder aber für jedes Element einer Liste. Dabei unterscheidet man zwischen \texttt{while}- und \texttt{for}-Schleifen. Allgemeine Schemata:
	\begin{lstlisting}[language=Python]
while CONDITION:
  # your code here #
	
for ELEMENT in LIST:
  # your code here #
	\end{lstlisting}
	Beispiele:
	\begin{lstlisting}[language=Python]
while True:
  mc.postToChat("Hallo Welt")
	
while x < 5:
  x = x + 1

for number in [1,2,3,4,5]:
  print(number)
	\end{lstlisting}
\end{document}