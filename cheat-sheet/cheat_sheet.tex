\documentclass{article}

\usepackage[english]{babel} % required to compile for windows
\usepackage[utf8]{inputenc}
\usepackage[T1]{fontenc}
\usepackage{geometry}
\usepackage{listings}
\usepackage{xcolor}
\lstset{ %
	frame=single,	                   % adds a frame around the code
	numbers=left,                    % where to put the line-numbers; possible values are (none, left, right)
	numbersep=5pt,                   % how far the line-numbers are from the code
	numberstyle=\tiny\color{gray}, % the style that is used for the line-numbers
	showstringspaces=false
}
\usepackage{enumitem}
\setlist[itemize]{leftmargin=*}
\usepackage{fancyhdr} % header
\geometry{a4paper, left=2cm, right=2cm, top=3cm, bottom=4cm} 

% header
\pagestyle{fancy}
\fancyhead{} % clear header
\fancyfoot{} % clear footer
\setlength{\headheight}{50pt}

\chead{{\large Schnupperuni Informatik}}

\lhead{FOSS-AG}
\rhead{23-26.10.2017}
\cfoot{-\thepage-}

\begin{document}
	\begin{center}
		{\huge Python Cheat Sheet}
	\end{center}

	\section{Variablen}
		In Variablen lassen sich beliebige Werte, aber auch Ergebnisse von Berechnung und Funktionen abspeichern.\\
		\textbf{Allgemeines Schema:}
		\begin{lstlisting}[language=Python]
variable_name = value
		\end{lstlisting}
		\textbf{Beispiele:}
		\begin{itemize}
			\item[] \begin{lstlisting}[language=Python]
text = "Hello World"
			\end{lstlisting}
			Speichert den String \texttt{Hello World} in einer Variablen mit Namen \texttt{text} ab.
			
			\item[] \begin{lstlisting}[language=Python]
x = 3
			\end{lstlisting}
			Speichert die Zahl 3 in einer Variablen mit Namen \texttt{x} ab.
			
			\item[] \begin{lstlisting}[language=Python]
a = x + 1
			\end{lstlisting}
			Speichert das Ergebnis der Rechnung \texttt{x + 1} in einer Variablen mit Namen \texttt{a}.
			
			\item[] \begin{lstlisting}[language=Python]
position = mc.player.getPos()
			\end{lstlisting}
			Speichert das Ergebnis der Funktion \texttt{getPos} in einer Variablen mit Namen \texttt{position}.
		\end{itemize}
	
	\section{Bedingungen}
	Manchmal möchte man einen bestimmten Codeabschnitt nur dann ausführen, wenn gewisse Bedingungen gelten. Genau für diese Fälle verwendet man \texttt{if-}Abfragen. Ist die Bedingung wahr, so wird der Abschnitt ausgeführt, andernfalls wird er einfach übersprungen oder der dazugehörige \texttt{else}-Block ausgeführt.\\
	\textbf{Allgemeines Schema:}
	\begin{lstlisting}[language=Python]
if CONDITION:
  # your code here
	\end{lstlisting}
	\textbf{Beispiele:}
	\begin{itemize}
		\item[] \begin{lstlisting}[language=Python, literate={==}{={}=}2]
if x == 10:
  x = x - 5
		\end{lstlisting}
		Falls \texttt{x} gleich 10 gilt, wird von \texttt{x} 5 subtrahiert, andernfalls passiert nichts.
		
		\item[] \begin{lstlisting}[language=Python]
if x > 0 and x < 3:
  x = 10
else:
  x = 20
		\end{lstlisting}
		Falls \texttt{x} größer als 0 und kleiner als 3 ist, wird \texttt{x} auf 10 gesetzt, andernfalls wird \texttt{x} auf 20 gesetzt.
	\end{itemize}
	
	\section{Schleifen}
	Mit Hilfe von Schleifen kann ein bestimmter Codeabschnitt mehrfach ausgeführt werden oder aber für jedes Element einer Liste. Dabei unterscheidet man zwischen \texttt{while}- und \texttt{for}-Schleifen.\\
	\textbf{Allgemeine Schemata:}
	\begin{lstlisting}[language=Python]
while CONDITION:
  # your code here
	
for ELEMENT in LIST:
  # your code here
	\end{lstlisting}
	\textbf{Beispiele:}
	\begin{itemize}
		\item[] \begin{lstlisting}[language=Python]
while True:
  mc.postToChat("Hello World")
		\end{lstlisting}
		Diese Schleife wird nicht stoppen, da \texttt{True} immer wahr ist (wörtliche Übersetzung). Also wird unendlich oft der String \texttt{Hello World} in den Minecraft-Chat geschrieben.
		
		\item[] \begin{lstlisting}[language=Python]
while x < 5:
  x = x + 1
		\end{lstlisting}
		Diese Schleife wird solange durchlaufen, wie \texttt{x} kleiner als 5 ist. In jedem Durchlauf wird \texttt{x} dann um 1 erhöht.
		
		\item[] \begin{lstlisting}[language=Python]
for number in [1,2,3,4,5]:
  print(number)
		\end{lstlisting}
		Diese Schleife wird einmal für jedes Element der Liste \texttt{[1,2,3,4,5]} durchlaufen. Jede Zahl der Liste wird also einmal ausgegeben.
	\end{itemize}

	\section{Nützliche Funktionen}
	\begin{itemize}
		\item[] \begin{lstlisting}[language=Python]
pos = mc.player.getPos()
		\end{lstlisting}
		liefert die aktuelle Koordinaten (Position) von Steve und speichert sie in \texttt{pos}.
		\item[] \begin{lstlisting}[language=Python]
(x, y, z) = mc.player.getPos()
		\end{lstlisting}
		speichert die einzelnen Komponenten der Position in \texttt{x}, \texttt{y}, \texttt{z}.
		\item[] \begin{lstlisting}[language=Python]
mc.setBlock(x, y, z, block)
		\end{lstlisting}
		setzt den mit \texttt{block} bezeichneten Block an die gewünschte Position \texttt{(x, y, z)}.
		
		\item[] \begin{lstlisting}[language=Python]
hits = mc.events.pollBlockHits()

hit.pos.x
hit.pos.y
hit.pos.z
		\end{lstlisting}
		liefert eine Liste \texttt{hits} mit den Positionen der geschlagenen Blöcke. Die einzelnen Koordinaten dieser Position bekommt man mit beispielsweise mit \texttt{hit.pos.x}.
		
		\item[] \begin{lstlisting}[language=Python]
block = mc.getBlock(x, y, z)
		\end{lstlisting}
		liefert den Block an Position \texttt{(x, y, z)}.
		
		\item[] \begin{lstlisting}[language=Python]
x = random.randint(lower, upper)
		\end{lstlisting}
		liefert euch eine zufällig ganze Zahl zwischen \texttt{lower} und \texttt{upper}.
		
		\item[] \begin{lstlisting}[language=Python]
index = list.index(x)
		\end{lstlisting}
		liefert die Stelle, an der \texttt{x} in der Liste steht. \textbf{Achtung! Die erste Position einer Liste wird nicht mit 1, sondern mit 0 bezeichnet. Beispiel:}
		\item[] \begin{lstlisting}[language=Python]
list = ['a', 'b', 'c', 'd']
		
list.index('a')  ->  0
list.index('c')  ->  2
		\end{lstlisting}
	\end{itemize}

\end{document}\grid
